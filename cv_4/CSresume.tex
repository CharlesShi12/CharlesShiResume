\documentclass[11pt]{res} 
\setlength{\textheight}{9.5in} 
\setlength{\textwidth}{\resumewidth}
\usepackage{geometry}
\usepackage{fontawesome}
\usepackage{hyperref}
\usepackage{enumitem}
\resumewidth = 7.34in 
\geometry{left=.58in, top=.47in, right=.58in, bottom=.49in}
\newsectionwidth{0pt}  
\linespread{1.1} 
\begin{document} 
\begin{footnotesize}
\moveleft\hoffset\centerline{\LARGE\bf Charles Shi} 
\moveleft\hoffset\vbox{\hrule width\resumewidth height 1pt}\smallskip
\moveleft\hoffset\centerline{shi46@illinois.edu \textbar{} (612) 986 - 0487 \textbar{}  Eden Prairie, MN}  
\moveleft\hoffset\centerline{\faLinkedinSquare \vspace{2mm}  \href{https://www.linkedin.com/in/cshi02/}{linkedin.com/in/cshi02} \textbar{} \faGithub \vspace{2mm} \href{https://github.com/CharlesShi12}{github.com/CharlesShi12} \textbar{} \faUser \vspace{2mm} \href{https://charlesshi12.github.io/}{charlesshi12.github.io}} \vspace{-5mm}
\begin{resume}
\vspace{-5mm}
\begin{small}
\section{EDUCATION}
\end{small}
\vspace{.5mm}
\textbf{University of Illinois at Urbana-Champaign} \hfill Expected May 2023 \\
{\sl Bachelor of Science,} Statistics and Computer Science \hfill GPA: 3.9/4.0

\begin{small}
\section{EXPERIENCE}
\end{small}
\vspace{.5mm}
\textbf{Futurist Academy} — {\sl Software Developer Intern} \hfill July 2020 - September 2020\vspace{-5mm}
\begin{itemize}[leftmargin=6.25mm]
\item Created a similarity search algorithm called MedSearch that takes in the abstract of any research paper and outputs similar COVID-19 research papers to advance COVID-19 research.
\end{itemize}
\vspace{-6.4mm}
\begin{itemize} \itemsep -2pt 
\item[$\circ$] Extracted keywords from over \textbf{125,000} COVID-19 research papers using Natural Language Processing.
\vspace{1mm}
\item[$\circ$] Stored each paper's ID and keywords in a TigerGraph graph database (stored over \textbf{350,000} nodes \& \textbf{1,000,000} edges).
\vspace{1mm}
\item[$\circ$] Wrote GSQL queries that found the most similar COVID-19 research papers using the Jaccard similarity index.
\vspace{1mm}
\item[$\circ$] Built and used a REST API for MedSearch's backend to safely interact with TigerGraph and enhance overall security.
\end{itemize}
\vspace{-6mm}
\begin{itemize}[leftmargin=6.25mm]
\item Developed a personalized patient dashboard that gives doctors and researchers an in-depth analysis of synthetic patient data through informative visualizations and statistics.
\end{itemize}
\vspace{-6.95mm}
\begin{itemize} \itemsep -2pt 
\item[$\circ$] Obtained Synthea-generated patient data and computed patient statistics by writing multiple GSQL queries in TigerGraph.
\vspace{1mm}
\item[$\circ$] Programmed the patient dashboard and data visualizations using Dash and Plotly in Python.
\end{itemize}
\vspace{-2.5mm}
\textbf{STEM Builders} — {\sl Computer Science and Robotics Teacher} \hfill September 2018 - Present\vspace{-5mm}
\begin{itemize}[leftmargin=6.25mm] \itemsep -2pt 
\item Taught programming languages (Python, Java, HTML/CSS, MIT App Inventor, Scratch) and robotics to \textbf{K-8} students. 
\vspace{1mm}
\item Successfully mentored more than \textbf{75} students and gained leadership experience after adapting to unforeseen circumstances. 
\vspace{1mm}
\item Designed/planned final projects that assessed the students' problem solving skills while incorporating their interests.
\end{itemize}
\begin{small}
\section{PROJECTS}
\end{small} 
\vspace{.5mm}
\href{https://github.com/CharlesShi12/CalcShare}{\textbf{CalcShare}} — {\sl JavaScript, HTML/CSS, Kotlin, Javalin, WebSockets, APIs} \hfill August 2020\vspace{-5mm}
\begin{itemize}[leftmargin=6.25mm] \itemsep -2pt 
\item Constructed an interactive web-based calculator that allows real-time collaboration between users. 
\vspace{1mm}
\item Utilized Kotlin to design a server with unique IDs for each room and data structures that store each room's collaborators and their previous calculations.
\vspace{1mm}
\item Developed a JavaScript client to communicate with the server, implemented WebSockets to handle the constant influx of data, and interacted with a RESTful API to compute equations inputted by the user. 
\end{itemize}
\vspace{-2.5mm}
\href{https://github.com/CharlesShi12/AI_Tumor_Scanner}{\textbf{Tumor Scanner}} — {\sl Python, OS Module, TensorFlow/Keras, Streamlit, Neural Network}\hfill July 2020\vspace{-5mm}
\begin{itemize}[leftmargin=6.25mm] \itemsep -2pt 
\item Collaborated with a team of three other developers to create a Convolutional Neural Network capable of identifying tumors from brain MRI scans.
\vspace{1mm}
\item Built the neural network with TensorFlow/Keras and used data augmentation to train it with more than \textbf{7,000} images.
\vspace{1mm}
\item Tested and modified the neural network to obtain a final average accuracy of \textbf{95\%}. 
\end{itemize}
\vspace{-2.5mm}
\href{https://github.com/CharlesShi12/GibberishGenerator}{\textbf{Gibberish Generator}} — {\sl Java, Trie Data Structure, Conditional Probability} \hfill May 2020\vspace{-5mm}
\begin{itemize}[leftmargin=6.25mm] \itemsep -2pt 
\item Programmed an algorithm in Java that generates random English-like words using models and optimized data structures. 
\vspace{1mm}
\item Trained the models with over \textbf{80,000} English words to produce accurate/pronounceable outputs and designed a Trie data structure for efficiency.
\end{itemize}
\vspace{-2.5mm}
\href{https://github.com/CharlesShi12/ImageFilters}{\textbf{Image Filtering System}} — {\sl Python, Machine Learning, PPM Image Formatting} \hfill March 2020\vspace{-5mm}
\begin{itemize}[leftmargin=6.25mm] \itemsep -2pt 
\item Implemented a Python program that uses unsupervised machine learning to filter/reduce an image down to however many core colors its users select.
\vspace{1mm}
\item Applied PPM raster image formatting and constructed a k-means clustering algorithm to find the dominant colors of an inputted image and filter/reduce the image based on those colors. 
\end{itemize}
\begin{small}
\section{SKILLS}
\end{small} 
\vspace{.5mm}
{\sl Languages:} 
Python, Java, HTML/CSS, JavaScript, Kotlin, GSQL \\
{\sl Frameworks/Technologies:} ReactJS, Flask, Streamlit, Dash, Javalin, Git, TigerGraph

\end{resume}
\end{footnotesize}
\end{document}
