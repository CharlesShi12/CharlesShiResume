\documentclass[11pt]{res} 
\setlength{\textheight}{9.5in} 
\setlength{\textwidth}{\resumewidth}
\usepackage{geometry}
\usepackage{fontawesome}
\resumewidth = 7.2in 
\geometry{left=.65in, top=.45in, right=.65in, bottom=.48in}
\newsectionwidth{0pt}  
\begin{document} 
\begin{footnotesize}
\moveleft\hoffset\centerline{\LARGE\bf Charles Shi} 
\moveleft\hoffset\vbox{\hrule width\resumewidth height 1pt}\smallskip
\moveleft\hoffset\centerline{shi46@illinois.edu \textbar{} 612 - 986 - 0487 \textbar{}  Eden Prairie, MN }  
\moveleft\hoffset\centerline{\faLinkedinSquare \vspace{2mm}  linkedin.com/in/cshi02 \textbar{} \faGithubSquare \vspace{2mm} github.com/CharlesShi12 \textbar{} \faUser \vspace{2mm} charlesshi12.github.io} \vspace{-5mm}
\begin{resume}
\vspace{-5mm}
\section{EDUCATION}
\textbf{University of Illinois at Urbana-Champaign} \hfill Expected May 2023 \\
{\sl Bachelor of Science,} Statistics and Computer Science \hfill GPA: 3.9/4.0
 
\section{EXPERIENCE}
\textbf{Futurist Academy} {\sl Software Developer Intern} \hfill July 2020 - September 2020\vspace{-4.5mm}

Responsible for designing impactful projects that utilize TigerGraph's graph database and presenting the finished projects to a group of businesses ranging from startups to Fortune 500 companies.
 
\vspace{-4mm}
\underline{MedSearch:}
\vspace{-5.10mm}
\begin{itemize} \itemsep -2pt 
\item Created MedSearch—a similarity search algorithm that takes in the abstract of any COVID-19 research paper and returns other similar/related COVID-19 research papers—to empower collaboration and advancement in COVID-19 research. 
\vspace{.75mm}
\item Extracted keywords from over \textbf{125,000} COVID-19 research papers using Natural Language Processing and stored each paper's ID and keywords in TigerGraph (stored over \textbf{350,000} nodes \& \textbf{1,000,000} edges).
\vspace{.75mm}
\item Wrote GSQL queries that found the most similar COVID-19 research papers using the NLP-extracted keywords, user-inputted abstract, and Jaccard similarity index.
\vspace{.75mm}
\item Built and used a RESTful API for MedSearch's backend to safely interact with TigerGraph and enhance overall security.
\end{itemize}
\vspace{-4.75mm}
\underline{Patient Dashboard:}
\vspace{-5.10mm}
\begin{itemize} \itemsep -2pt 
\item Developed a personalized patient dashboard that gives doctors and researchers an in-depth analysis of synthetic patient data through informative visualizations and statistics. 
\vspace{.75mm}
\item Obtained Synthea-generated patient data and computed patient statistics by writing multiple GSQL queries in TigerGraph.
\vspace{-3mm}
\item Programmed the patient dashboard and data visualizations using Dash and Plotly.
\end{itemize}
\vspace{-2mm}
\textbf{STEM Builders} {\sl Computer Science and Robotics Teacher} \hfill September 2018 - Present\vspace{-5.25mm}
\begin{itemize} \itemsep -2pt 
\item Taught various programming languages (Python, HTML/CSS, MIT App Inventor, Scratch) and robotics to K-8 students. 
\vspace{.75mm}
\item Created custom learning curricula and designed/planned final projects that assessed the students' problem solving skills while incorporating their interests and curiosities. 
\vspace{.75mm}
\item Monitored the students' progress and provided daily feedback to their parents.
\end{itemize}
\section{PROJECTS} 
\textbf{Real-Time Collaborative Calculator} \hfill August 2020\vspace{-5.25mm}
\begin{itemize} \itemsep -2pt 
\item Designed an interactive web-based calculator that allows real-time collaboration between users. 
\vspace{.75mm}
\item Utilized Kotlin to develop a server with unique IDs for each room and data structures that store each room's collaborators and their previous calculations.
\vspace{.75mm}
\item Constructed a JavaScript client to communicate with the server, implemented WebSockets to effectively deal with the constant influx of data, and interacted with a RESTful API to compute equations inputted by the user.  
\end{itemize}
\vspace{-2.5mm}
\textbf{AI Tumor Scanner}  \hfill July 2020\vspace{-5.25mm}
\begin{itemize} \itemsep -2pt 
\item Collaborated with a team of three other developers to create a Convolutional Neural Network capable of identifying tumors from brain MRI scans. 
\vspace{.75mm}
\item Built the neural network with TensorFlow/Keras and used data augmentation to train it with over \textbf{7,000} images.
\vspace{.75mm}
\item Tested and modified the neural network to obtain a final average accuracy of \textbf{95\%}.
\end{itemize}
\vspace{-2.5mm}
\textbf{Gibberish Generator} \hfill May 2020\vspace{-5.25mm}
\begin{itemize} \itemsep -2pt 
\item Programmed an algorithm in Java that generates random English-like words using a trained model and highly optimized data structures.  
\vspace{.75mm}
\item Trained the model with over \textbf{80,000} English words to produce accurate/pronounceable outputs and developed a Trie data structure for efficiency.
\end{itemize}
\vspace{-2.5mm}
\textbf{Image Filtering System} \hfill March 2020\vspace{-5.25mm}
\begin{itemize} \itemsep -2pt
\item Implemented a Python program that uses unsupervised machine learning to filter/reduce an image down to however many core colors its users select.
\vspace{.75mm}
\item Applied PPM raster image formatting and k-means clustering to group the pixels of an inputted image into its dominant color clusters and filter/reduce the image based on those clusters. 
\end{itemize}
\section{SKILLS} 
{\sl Languages:} 
Python, Java, HTML/CSS, JavaScript, Kotlin, GSQL, \LaTeX \\
{\sl Frameworks/Technologies:} ReactJS, Flask, Streamlit, Dash, Git, TigerGraph, Firebase

\end{resume}
\end{footnotesize}
\end{document}
